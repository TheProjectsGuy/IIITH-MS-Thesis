\documentclass{article}
\usepackage[a4paper, textwidth=160mm, textheight=220mm]{geometry}
\usepackage[sorting=ydnt]{biblatex}
\usepackage{hyperref}
\usepackage{titlesec}
\usepackage{setspace}
\usepackage{amssymb}
\usepackage{lipsum}
\usepackage[en-IN]{datetime2}

% Configurations
% \setlength{\parindent}{0pt}
\pagestyle{empty}
\onehalfspacing
\addbibresource{bibliography.bib}


\begin{document}
    \begin{center}
        {\bfseries\Large Thesis Synopsis} \\ [2.5mm]
        {\large Foundation Models for Visual Place Recognition}
    \end{center}
    {\bf Name:} Avneesh Mishra
    \hfill {\bf Roll No.:} 2021701032
    \hfill {\bf Date:} \today \\ [2mm]
    {\bf Advisor(s):} Prof. K. Madhava Krishna \\ [2mm]
    {\bf Department:} Robotics Research Center (RRC), KCIS, IIIT Hyderabad
    \vspace{2mm}
    \hrule
    \begin{center}
    \subsection*{\centering Abstract}
    \begin{minipage}{0.85\textwidth}
        Imagine a robot navigating a complex and unfamiliar
        environment - underwater, subterranean, or even a dilapidated
        building. Current Visual Place Recognition (VPR) techniques
        often struggle with such diverse scenarios, requiring
        re-training for each new environment. This limitation hinders
        the development of truly autonomous robots. This work
        introduces AnyLoc, a novel VPR system taking a significant
        step towards universality. AnyLoc leverages feature
        representations learned by powerful foundation models,
        eliminating the need for VPR-specific training. Furthermore,
        by combining these features with unsupervised aggregation
        techniques, AnyLoc can uncover unique visual characteristics
        that define different environments. Our experiments
        demonstrate AnyLoc’s potential to function across multiple
        environments (outdoor, indoor, aerial, underwater,
        subterranean, and dilapidated) without retraining, laying the
        groundwork for VPR solutions that could be deployed anywhere,
        anytime, and across anyview. \\ [2mm]
        {\bf Keywords:} Visual Place Recognition (VPR), Foundation 
        Models (FM), Localization using Deep Learning
    \end{minipage} \\ [2mm]
    \end{center}

    \subsection*{Contribution}
    This thesis introduces AnyLoc \cite{Keetha2023AnyLocTU}, a novel
    method for generic image retrieval (IR) in the context of
    localization tasks. AnyLoc leverages features learned by
    foundation models (FMs) like DINOv2 \cite{Oquab2023DINOv2LR} to
    extract global image descriptors. These descriptors are effective
    across diverse environments (indoor, outdoor, aerial,
    subterranean, degraded, and underwater) and scene variations
    (day/night, viewpoint changes). Our approach utilizes latent
    features from the penultimate layers of foundation models,
    achieving state-of-the-art performance. We employ established
    local descriptor aggregation techniques (VLAD
    \cite{Arandjelovi2013AllAV}, GeM pooling
    \cite{Radenovic2017FineTuningCI}) on patch descriptors to further
    improve retrieval accuracy. Furthermore, we demonstrate that
    applying dimensionality reduction methods like PCA significantly
    reduces the memory footprint (up to 95x) without sacrificing
    performance. This offers a valuable advantage for
    resource-constrained deployment scenarios. Interestingly, we
    discover that FMs inherently learn domains (like indoor, aerial,
    etc.) from the data. This allows us to create domain-specific
    off-the-shelf VPR systems that outperform supervised counterparts
    by a significant margin. We also present an early benchmark on the
    performance of various FMs for VPR across diverse environments.

    More specifically, this thesis introduces two novel image
    retrieval methods for localization tasks: AnyLoc-VLAD-DINO and
    AnyLoc-VLAD-DINOv2. Both leverage foundation models like DINO and
    DINOv2 to extract powerful image descriptors. To enhance
    performance across diverse environments, we create domain-specific
    retrieval models. We achieve this by aggregating features from
    images within similar domains (e.g., urban, indoor, aerial)
    identified using a dimensionality reduction technique. Within the
    FMs, we strategically select specific layers and features that
    best capture these domain-specific properties. The selection
    process and the insights gained from analyzing different features
    are further elaborated upon in the thesis.

    Our presented VLAD models beat supervised VPR baselines by over
    5\% for indoor datasets, and by over 47\% for aerial datasets. We
    get comparable performance even when comparing on large outdoor
    datasets (94\% against 95\% for baseline), demonstrating their
    generalizability across diverse environments. These models are
    especially beneficial to use in cases of unstructured datasets
    (like aerial, underwater, subterranean, and dilapidated
    buildings). Furthermore, even after undergoing a 96 times
    reduction (from 49k to 512 dimension) using PCA, there is no
    significant performance drop (only 1\% on outdoor datasets). Using
    our models off-the-shelf, we hope this improves VPR and
    localization systems to work anytime, anywhere \& under anyview.

    \subsection*{Thesis Layout}
    In order to understand the work fully, the reader needs prior
    knowledge in two areas: visual place recognition (\emph{VPR})
    through global descriptors and vision foundation models
    (\emph{FM}). \textbf{Chapter 1} describes the contribution area
    from the perspective of mobile robotics. It gives an overview of
    Simultaneous Localization and Mapping (SLAM) systems and describes
    VPR techniques (from the perspective of global image descriptors
    through aggregation of local feature descriptors). \textbf{Chapter
    2} contains the background for foundation models. It elaborates on
    vision transformer models (ViT, DeIT, and CCT) and self-supervised
    learning (SSL) methods used for training DINO and DINOv2.
    \textbf{Chapter 3} presents AnyLoc, the core contribution of this
    thesis. It includes a comprehensive evaluation against supervised
    baselines and other foundation models. Additionally, ablation
    studies (quantitative and qualitative) provide insights into the
    model selection criteria. \textbf{Chapter 4} concludes the thesis
    with limitations and suggestions for future development.

    \vspace{3mm}
    \hrule
    \vspace{2mm}
    % \pagebreak
    \printbibliography
\end{document}
